\chapter{PENDAHULUAN}
    \section{Latar Belakang Masalah}

    Hakikat kehidupan di dunia adalah tidak lepas akan ujian dari Allah SWT. Allah SWT memberikan ujian berupa kesenangan dan kesusahan. Berdasarkan hal tersebut, diperlukan rasa syukur dan sabar. Seseorang yang beruntung akan mendapatkan pahala orang yang bersyukur dalam kondisi senang dan pahala orang bersabar dalam kondisi susah. Sebagaimana firman Allah SWT dalam surah Ibrahim ayat 5, Luqman ayat 31, Saba’ ayat 19, dan Asy-Syu’ara ayat 33 yang berbunyi:

    \begin{flushright}
        \<اِنَّ فِيْ ذٰلِكَ لَاٰيٰتٍ لِّكُلِّ صَبَّارٍ شَكُوْرٍ (٥)>
    \end{flushright}

    Artinya: “Sesungguhnya dalam yang demikian itu terdapat tanda-tanda bagi orang yang bersabar dan bersyukur”. Allah SWT memutuskan ketetapan bagi manusia kecuali hal itu baik baginya \citep{Ramdhan2019}. Seperti kisah Nabi Musa AS. dan kaumnya ketika Allah SWT selamatkan mereka dari Fir’aun dan siksaan yang menghinakan. Berdasarkan hal tersebut, terdapat pelajaran yang besar bagi setiap orang yang bersabar menghadapi kesengsaraan dan bersyukur dalam kenikmatan \citep{Muaziroh2018}. Sebaik-baik hamba ialah orang yang bersabar ketika mendapatkan cobaan dan bersyukur ketika mendapatkan kenikmatan. Sebagaimana kisah Nabi Ayyub AS. ketika beliau menderita penyakit kulit, yaitu kusta atau lepra. Allah SWT menyebutkan kisah Nabi Ayyub AS. dalam surah Al-Anbiya’ ayat 83-84:

    \begin{flushright}
        \begin{RLtext}
            وَاَيُّوْبَ اِذْ نَادٰى رَبَّهٗٓ اَنِّيْ مَسَّنِيَ الضُّرُّ وَاَنْتَ اَرْحَمُ الرّٰحِمِيْنَ ۚ (٣٨) فَاسْتَجَبْنَا لَهٗ فَكَشَفْنَا مَا بِهٖ مِنْ ضُرٍّ وَّاٰتَيْنٰهُ اَهْلَهٗ وَمِثْلَهُمْ مَّعَهُمْ رَحْمَةً مِّنْ عِنْدِنَا وَذِكْرٰى لِلْعٰبِدِيْنَ ۚ (٤٨)
        \end{RLtext}
    \end{flushright}

    Artinya: “Dan (ingatlah kisah) Ayub, ketika ia menyeru Rabbnya: “(Ya Rabbku), sesungguhnya aku telah ditimpa penyakit dan Engkau adalah Rabb Yang Maha Penyayang di antara semua penyayang.” Maka Kamipun memperkenankan seruannya itu, lalu Kami lenyapkan penyakit yang ada padanya dan Kami kembalikan keluarganya kepadanya, dan Kami lipat gandakan bilangan mereka, sebagai suatu rahmat dari sisi Kami dan untuk menjadi peringatan bagi semua yang menyembah Allah”. Nabi Ayyub AS. diuji dengan penyakit kulit sehingga banyak orang menjauhinya. Salah satu penyakit kulit yang umum dan dapat menyebabkan kematian adalah kanker kulit \citep{Nurlitasari2022}. Selain kanker serviks dan kanker payudara, kanker kulit sering ditemukan di Indonesia. Terdapat sekitar 7.8\% kasus kanker kulit per tahun di Indonesia. Kasus kanker kulit yang sering terjadi di Indonesia ialah \textit{Basal Cell Carcinoma} (BCC) sekitar 65.5\%, \textit{Squamous Cell Carcinoma} (SCC) sekitar 23\%, dan \textit{Melanoma} (MEL) sekitar 7.9\%. Meskipun kasus BCC lebih sedikit daripada kasus MEL, akan tetapi jumlah kematian akibat MEL 75\% dari keseluruhan jenis kanker kulit \citep{Fuadah2020a}. MEL merupakan kanker kulit yang paling berbahaya karena dapat menyebabkan kematian. Salah satu penyebab yang menjadikan MEL sangat berbahaya adalah kemampuannya untuk menyebar ke dalam organ lain seperti jantung, hati, dan paru-paru \citep{Nugroho2019}.
    
    Kanker kulit merupakan kondisi pertumbuhan sel yang tidak normal pada kulit. Pertumbuhan sel yang tidak normal akan terus berjalan seiring waktu. Hal ini terjadi karena sering terpapar radiasi sinar \textit{Ultraviolet} (UV) secara langsung. Radiasi sinar UV dapat menembus lapisan luar kulit dan lapisan dalam kulit sehingga dapat merusak sel kulit termasuk sel \textit{Deoxyribonucleic acid} (DNA). DNA yang rusak dapat memicu kesalahan fungsi DNA sehingga DNA bermutasi. Akhirnya, hal ini dapat mengakibatkan pertumbuhan sel yang tidak terkontrol dan disebut kanker kulit \citep{Bhimavarapu2022}. Dampak kanker kulit dapat dikurangi dengan melakukan deteksi pada tahap awal karena morbiditas, mortalitas, dan biaya pengobatan yang sangat berkorelasi dengan tahap penyakit yang diderita. Tahap awal kanker kulit MEL masih merepresentasikan tingkat kematian yang tinggi. Kemudian, tahap lanjut dari MEL akan memperlihatkan gejala yang sangat buruk dan membutuhkan biaya perawatan yang lebih tinggi terutama pengobatan dengan imunoterapi \citep{Janda2022}. Terdapat dua jenis kanker kulit, yaitu \textit{benign} dan \textit{malignant}. \textit{Benign} merupakan kanker kulit yang tidak berbahaya sedangkan \textit{malignant} perlu penanganan khusus bahkan dapat menyebabkan kematian. Sebagian besar \textit{benign} tidak menyebabkan kematian karena \textit{benign} tidak menyebar ke dalam jaringan tubuh lain dan tidak tumbuh di tempat yang sama setelah dihilangkan sehingga \textit{benign} dapat dihilangkan dengan berbagai penanganan medis. Tahi lalat merupakan salah satu contoh dari \textit{benign}. Di sisi lain, \textit{malignant} memiliki tingkat kasus kematian yang cenderung tinggi. \textit{Malignant} mampu bertumbuh secara terus-menerus dan menyebar ke dalam jaringan di sekitarnya. MEL merupakan salah satu contoh dari \textit{malignant}. MEL merupakan jenis paling mematikan dari kanker kulit dan menyumbang 75\% dari kematian akibat kanker kulit. Pada tahap awal, MEL dapat ditangani dengan melakukan pembedahan sehingga pasien memiliki harapan hidup yang tinggi. Akan tetapi, pasien memiliki angka harapan hidup yang rendah jika sudah metastasis \citep{Davis2019}.

    Kanker kulit yang terjadi pada seseorang dapat diidentifikasi melalui gejala yang timbul pada kanker kulit itu sendiri. Gejala kanker kulit biasa disebut sebagai ABCDE oleh para dokter. ABCDE merupakan kependekan dari \textit{Asymmetry}, \textit{Border}, \textit{Color}, \textit{Diameter}, and \textit{Evolve} \citep{Gavrilov2019a}. \textit{Asymmetry} berarti tidak simetris yang menggambarkan terkait bentuk yang muncul pada kanker kulit. \textit{Border} berarti tepi yang merepresentasikan tepian tidak rata, bertekstur kabur, dan kasar pada kanker kulit. \textit{Color} berarti warna yang memperlihatkan kombinasi warna tidak proposional pada kanker kulit. Dengan kata lain, banyak kombinasi yang tidak teratur seperti hitam, coklat, abu-abu, dan merah. Diameter memperhitungkan kisaran diameter pada kanker kulit sekitar 6 mm sampai 0.25 inci. \textit{Evolve} menunjukkan terjadinya perubahan pada sel kanker selama terinfeksi kanker kulit \citep{Saherish2020a}. Dengan fitur ABCDE ini, dokter dapat melihat secara langsung untuk mendeteksi kanker kulit sejak tahap awal. Namun, cara tersebut kurang efisien karena mempertimbangkan penglihatan seseorang yang bisa saja berbeda. Sehingga dikembangkan sistem yang dinamakan \textit{Computer Aided Diagnosys} (CAD) untuk mendeteksi kanker kulit oleh pihak medis. CAD memiliki beberapa tahapan seperti memroses citra digital, ekstraksi fitur, dan klasifikasi \citep{Adyanti2017}.

    Terdapat beberapa penelitian tentang klasifikasi kanker kulit berdasarkan data citra dermoskopi sebelumnya. Ma dan Karki mengimplementasikan \textit{Machine Learning} (ML) untuk mendeteksi \textit{benign} dan \textit{malignant} dengan memanfaatkan fitur ABCD. Ma dan Karki mendapatkan akurasi klasifikasi yang cukup tinggi, yaitu $97.8\%$ menggunakan \textit{Support Vector Machine} (SVM) \citep{Ma2020}. Meskipun ML memiliki akurasi yang cukup tinggi pada beberapa kasus, ML tidak efektif dalam kasus diagnosis yang kompleks dalam praktik klinis. Pada umumnya, metode ML untuk mendeteksi kanker kulit terdiri dari ekstraksi fitur dan klasifikasi terhadap fitur yang sudah diekstraksi. Berdasarkan hal tersebut, ML memiliki keterbatasan pada fitur yang dipakai dan tidak efektif dalam studi kasus yang lebih luas \citep{Wu2022}.

    Alasadi dan Alsafy memanfaatkan pemrosesan citra digital untuk mendeteksi kanker kulit MEL berdasarkan fitur warna, bentuk, dan tekstur sehingga menghasilkan akurasi $98\%$ untuk mengklasifikasikan kanker kulit (\textit{benign} dan \textit{malignant}) dan $93\%$ untuk mengenali tipe MEL. Penelitian Alasadi dan Alsafy menggunakan \textit{Neural Network} (NN) untuk melakukan pengenalan tipe melanoma \citep{Alasadi2015a}. Salah satu perkembangan dari NN adalah \textit{Convolutional Neural Network} (CNN). CNN merupakan bagian dari \textit{deep learning} yang berguna untuk mengolah data citra sehingga CNN dapat mengklasifikasikan citra tanpa melakukan ekstraksi fitur sebelumnya. Karena pada CNN sudah terdapat proses ekstraksi fitur dan pembelajaran fitur. Pada CNN, terdapat berbagai arsitektur, seperti AlexNet, VGG, ResNet, GoogleNet, RCNN, Fast-RCNN, YOLO, dan masih banyak lagi. Nugroho, dkk. melakukan deteksi kanker kulit berdasarkan citra dermoskopi menggunakan CNN sehingga mendapatkan akurasi 78\%. Akurasi yang didapatkan oleh Nugroho, dkk. termasuk baik dengan mempertimbangkan tujuh kelas yang ada pada dataset yang digunakan oleh Nugroho, dkk. dalam penelitiannya \citep{Nugroho2019}. Pada penelitian yang dilakukan Fu’adah, dkk. yang menggunakan CNN, terdapat empat kelas untuk klasifikasi kanker kulit sehingga didapatkan nilai akurasi sebesar 99\% \citep{Fuadah2020a}.

    \textit{You Only Look Once} (YOLO) merupakan salah satu model deteksi objek \textit{real-time} berdasarkan CNN yang dipublikasikan pada tahun 2016 dengan \textit{mean Average Precision} (mAP) sebagai metrik evaluasi. YOLO-v1 mampu mendeteksi objek secara \textit{real-time} namun masih memiliki beberapa batasan, yaitu tidak dapat mendeteksi objek kecil dan akurasi yang kurang baik. YOLO-v1 mendapatkan $63.4$ nilai mAP pada dataset PASCAL VOC 2007 dan $57.9$ nilai mAP pada dataset PASCAL VOC 2012 \citep{Redmon2016a}. YOLO-v2 memperbaiki kelemahan YOLO-v1 dengan menambahkan \textit{batch normalization} untuk meningkatkan mAP dan menggunakan DarkNet-19 untuk memprediksi objek lebih dari satu kelas pada satu sel. DarkNet-19 terdiri dari 19 \textit{convolutional layers} dan 5 \textit{max-pooling layers}. YOLO-v2 juga dapat mendeteksi lebih dari 9000 kelas meskipun berdampak pada mAP. YOLO-v2 mendapatkan $73.4$ nilai mAP pada dataset PASCAL VOC 2012 dan $44$ nilai mAP pada dataset \textit{Microsoft Common Objects in Context} (MS COCO) \citep{Redmon2017}. YOLO-v3 mengganti DarkNet-19 dengan DarkNet-53 disertai \textit{residual blocks} dan \textit{upsampling networks}. Kelebihan YOLO-v3 dapat memprediksi 3 ukuran yang berbeda. YOLO-v3 mendapatka $57.9$ nilai mAP pada dataset MS COCO \citep{Redmon2018}. YOLO-v4 menambahkan \textit{Weighted Residual Connections}, \textit{Cross Mini Batch Normalization}, \textit{Cross Stage Partial Connections}, \textit{Self Adversarial Training}, dan \textit{Mish Activation}. YOLO-v4 mendapatkan $65.7$ nilai mAP pada dataset MS COCO \citep{Bochkovskiy2020}.

    Banyak sekali penelitian sebelumnya yang menggunakan YOLO karena kemampuannya yang dapat mendeteksi objek secara \textit{real-time}. Bahkan, per tahun 2022 sudah terdapat YOLO-v7 meskipun terdapat banyak sekali versi YOLO yang tidak resmi. Kemunculan versi YOLO yang tidak resmi tersebut dikarenakan banyaknya penggemar YOLO yang ingin memodifikasi dan membuat versi YOLO sendiri. Berdasarkan kemunculan banyak versi YOLO sebelumnya, YOLO memiliki berbagai versi yang terus dan masih dikembangkan hingga saat ini. Chhatlani, dkk. melakukan deteksi \textit{melanoma} dan bukan \textit{melanoma} menggunakan YOLOv5 sehingga menghasilkan \textit{Average Precision} (AP) sebesar 89\% untuk kedua kelas, 93\% untuk \textit{melanoma}, dan 83\% untuk bukan \textit{melanoma} \citep{Chhatlani2022a}.
    
    Metode YOLO sudah terbukti dan seringkali menjadi metode yang disarankan untuk melakukan deteksi objek. Perkembangan YOLO dari tahun 2016 hingga tahun 2020 memunculkan banyak variasi dan modifikasi. YOLO sangat bergantung pada pengolahan data saat pembentukan model. Misalnya, ukuran data masukan yang terlalu besar membuat pembentukan model menghabiskan banyak waktu. Di sisi lain, ukuran data masukan yang terlalu kecil membuat YOLO kesulitan untuk mengenali objek pada citra. Sehingga, diperlukan proses \textit{resize} yang tepat untuk menentukan ukuran citra pada proses pembentukan model.
    
    % TODO: Revisi kalimat dengan menyebutkan penelitian siapa saja yang berhasil menggunakan YOLO
    % TODO: Tambahkan keuntungan menggunakan YOLO-v7-Tiny
    Berdasarkan penelitian-penelitian yang sudah ada sebelumnya, algoritma YOLO sangat cepat dan akurat untuk melakukan tugas deteksi objek sehingga penelitian ini menggunakan YOLO-v7 untuk melakukan deteksi kanker kulit berdasarkan citra dermoskopi. Penelitian ini memanfaatkan proses \textit{resize} untuk mengurangi waktu komputasi dan menemukan ukuran citra yang sesuai untuk pembentukan model deteksi kanker kulit. Penelitian ini diharapkan dapat membuat sistem diagnosis kanker kulit menggunakan YOLO-v7 sehingga terdapat alternatif pendeteksian kanker kulit yang efisien.

    \section{Rumusan Masalah}
    Berdasarkan penguraian masalah pada latar belakang penelitian, terdapat rumusan masalah sebagai berikut:
    \begin{enumerate}
        \item Bagaimana proses deteksi kanker kulit menggunakan YOLO-v7 berdasarkan deteksi objek pada data citra dermoskopi?
        \item Bagaimana hasil deteksi kanker kulit menggunakan YOLO-v7 berdasarkan uji coba \textit{batch size}, YOLO-v7, dan YOLO-v7-Tiny?
    \end{enumerate}

    \section{Tujuan Penelitian}
    Berdasarkan kedua rumusan masalah di atas, penelitian ini dilakukan dengan tujuan sebagai berikut:
    \begin{enumerate}
        \item Mengetahui proses deteksi kanker kulit menggunakan YOLO-v7 berdasarkan deteksi objek pada data citra dermoskopi.
        \item Mengetahui hasil deteksi kanker kulit menggunakan YOLO-v7 berdasarkan uji coba \textit{batch size}, YOLO-v7, dan YOLO-v7-Tiny?
    \end{enumerate}

    \section{Manfaat Penelitian}
    Penelitian ini diharapkan dapat memberikan manfaat pada seluruh pihak, seperti yang dipaparkan berikut ini:
    \begin{enumerate}
        \item Manfaat Teoritis
        Penelitian ini diharapkan dapat menjadi referensi untuk para peneliti berikutnya dalam deteksi kanker kulit menggunakan YOLO-v7 berdasarkan deteksi objek pada data citra dermoskopi.

        \item Manfaat Praktis
        \begin{enumerate}
            \item Bagi Penulis
            Meningkatkan pengetahuan penulis dalam menerapkan algoritma YOLO untuk deteksi kanker kulit pada citra dermoskopi.

            \item Bagi Tim Medis
            Meningkatkan efisiensi tim medis untuk mendeteksi kanker kulit dengan kemudahan dan keakuratan yang lebih signifikan.

            \item Bagi Masyarakat
            Meningkatkan waktu diagnosis kanker kulit dan memberikan bahan edukasi kepada masyarakat terkait tingkat bahaya kanker kulit.
        \end{enumerate}
    \end{enumerate}

    \section{Batasan Masalah}
    Mempertimbangkan ruang lingkup permasalahan yang luas, maka penelitian ini menggunakan batasan-batasan masalah sebagai berikut:
    \begin{enumerate}
        \item Metode deep learning yang digunakan pada penelitian ini untuk mendeteksi kanker kulit menggunakan data citra dermoskopi adalah metode YOLO-v7.
        \item Data citra yang digunakan pada penelitian ini untuk mendeteksi kanker kulit adalah data citra dermoskopi.
        \item Kategori kanker yang digunakan pada penelitian ini yaitu \textit{melanoma}, \textit{actinic keratosis}, \textit{nevus}, \textit{basal cell carcinoma}, \textit{squamous cell carcinoma}, \textit{dermatofibroma}, \textit{benign keratosis lesion}, dan \textit{vascular lesion}.
    \end{enumerate}

    \section{Sistematika Penulisan}
    Penelitian ini tersusun atas lima bab yang memuat seluruh isi penelitian dan diringkas pada sistematika penulisan sebagai berikut:
    \begin{enumerate}
        \item BAB I PENDAHULUAN
        memaparkan tentang latar belakang penelitian, rumusan masalah penelitian, tujuan penelitian, manfaat penelitian, dan sistematika penulisan penelitian.
        \item BAB II TINJAUAN PUSTAKA
        memaparkan tentang teori-teori yang digunakan pada penelitian ini berdasarkan jurnal dan buku yang mendukung penelitian ini. Tinjauan pustaka penelitian ini memuat teori tentang kanker kulit dan citra dermoskopi, metode pada tahap pre-processing menggunakan \textit{annotation} dan \textit{resize}, metode pada tahap deteksi kanker kulit menggunakan \textit{You Only Look Once} (YOLO), dan metode pada tahap evaluasi menggunakan \textit{mean Average Precision}.
        \item BAB III METODE PENELITIAN 
        memaparkan tentang proses memperoleh data dan mengolah data sehingga rumusan masalah pada penelitian ini dapat terselesaikan.
        \item BAB IV HASIL DAN PEMBAHASAN
        memaparkan tentang hasil penelitian terkait proses yang terjadi pada deteksi kanker kulit dan analisis hasil yang diperoleh.
        \item BAB V PENUTUP
        memaparkan tentang kesimpulan pada penelitian ini dan saran penulis pada penelitian selanjutnya.
    \end{enumerate}